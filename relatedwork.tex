
\section{Related Work}\label{sec:rw}

In a recommender system, there is a set of users and a set of items. The goal of the system is to recommend a user certain items that best match the user's preference. The essential research problem here is, how to predict a user's rating for items that were not rated by him. This is due to the fact that the number of users and items are usually very large, and it is impractical to ask users to rate every item. We study this problem in the context of user-site recommendations.

Two types of recommender systems have been developed in the past few decades. \textbf{Content-based recommender systems} (e.g., \cite{contentbasedLang95,contentbasedPazzani97}) analyze properties of item (e.g., item descriptions) and/or user profile to identify items that are attractive to the user. The spatial-temporal features or visiting patterns we explored is similar to this type of recommendation. Nevertheless, existing methods mostly focus on features extracted from user's interaction with the recommender system (e.g., view or purchase history, reviews, user profile).

The other type is \textbf{Collaborative-filtering recommender systems} (e.g., \cite{userUserRec94,amazonRecommendation,MFRec09})). Collaborative-filtering predicts users' preference to items based on their similarity to other users. It relies on analysis of large amount existing product rating data. However, it becomes challenging to calculate user-similarity when there is not enough such data. This is known as the cold-start problem. A widely used technique for collaborative-filtering is Matrix factorization~\cite{koren2009matrix}. Matrix factorization works on a user-item rating matrix. It models both users and items as vectors of latent features.

Our work is different from existing Location Recommendation in LBSN~\cite{yu2015survey, ye2010location,wang2013location, cheng2012fused}, which also considers PoI as recommended items. These works propose to explore geo-social activities of users to facilitate the identification of users with similar preference in matrix factorization. For example, users demonstrate similar location-visiting pattern is likely to have similar rating for PoIs. Most works in this category combines check in data with additional information (e.g., social connection, user profile). These techniques fall in the category of collaborative-filtering. In contrast, our work is to direct predict a user's preference purely based on explicit features extracted from his trajectory, which is a fundamentally different approach and was not investigated in literature to our knowledge.