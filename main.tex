% This is "sig-alternate.tex" V2.1 April 2013
% This file should be compiled with V2.5 of "sig-alternate.cls" May 2012
%
% This example file demonstrates the use of the 'sig-alternate.cls'
% V2.5 LaTeX2e document class file. It is for those submitting
% articles to ACM Conference Proceedings WHO DO NOT WISH TO
% STRICTLY ADHERE TO THE SIGS (PUBS-BOARD-ENDORSED) STYLE.
% The 'sig-alternate.cls' file will produce a similar-looking,
% albeit, 'tighter' paper resulting in, invariably, fewer pages.
%
% ----------------------------------------------------------------------------------------------------------------
% This .tex file (and associated .cls V2.5) produces:
%       1) The Permission Statement
%       2) The Conference (location) Info information
%       3) The Copyright Line with ACM data
%       4) NO page numbers
%
% as against the acm_proc_article-sp.cls file which
% DOES NOT produce 1) thru' 3) above.
%
% Using 'sig-alternate.cls' you have control, however, from within
% the source .tex file, over both the CopyrightYear
% (defaulted to 200X) and the ACM Copyright Data
% (defaulted to X-XXXXX-XX-X/XX/XX).
% e.g.
% \CopyrightYear{2007} will cause 2007 to appear in the copyright line.
% \crdata{0-12345-67-8/90/12} will cause 0-12345-67-8/90/12 to appear in the copyright line.
%
% ---------------------------------------------------------------------------------------------------------------
% This .tex source is an example which *does* use
% the .bib file (from which the .bbl file % is produced).
% REMEMBER HOWEVER: After having produced the .bbl file,
% and prior to final submission, you *NEED* to 'insert'
% your .bbl file into your source .tex file so as to provide
% ONE 'self-contained' source file.
%
% ================= IF YOU HAVE QUESTIONS =======================
% Questions regarding the SIGS styles, SIGS policies and
% procedures, Conferences etc. should be sent to
% Adrienne Griscti (griscti@acm.org)
%
% Technical questions _only_ to
% Gerald Murray (murray@hq.acm.org)
% ===============================================================
%
% For tracking purposes - this is V2.0 - May 2012

%\documentclass{sig-alternate-05-2015}
\documentclass[conference]{IEEEtran}

\usepackage{amsmath}
\usepackage{url}



%\usepackage{times}
%\usepackage{mathtools}
\usepackage{multirow}
\usepackage{graphicx}
\usepackage{algorithm}
\usepackage{algorithmic}
\usepackage{color}
\usepackage{caption}
\usepackage{paralist}
\DeclareCaptionType{copyrightbox}
%\usepackage{subcaption}
\newtheorem{definition}{Definition}
\usepackage{amsmath}
\usepackage{url}
\usepackage{algorithm}
\usepackage{algorithmic}
\usepackage{booktabs}
\usepackage{amssymb}


%\usepackage{subfigure}
\usepackage[skip=1pt]{subcaption}

\usepackage{graphics}


\graphicspath{ {figures/} }

\newcommand{\myparatight}[1]{\smallskip\noindent{\bf {#1}:}~}

\renewcommand{\algorithmicrequire}{\textbf{Input:}}
\renewcommand{\algorithmicensure}{\textbf{Output:}}
\newcommand{\todo}[1]{{\textcolor{red}{{\bf TODO:} #1}}}
\newcommand{\tabincell}[2]{\begin{tabular}{@{}#1@{}}#2\end{tabular}}

\newcommand{\DB}{\ensuremath{\mathcal{ST}}}
\newcommand{\RR}{\ensuremath{\mathbb{R}}}
\newcommand{\NN}{\ensuremath{\mathbb{N}}}
\newcommand{\clust}{\ensuremath{\#\mbox{Clust}}}

\newcommand{\TODO}[1]{{\bf \textcolor[rgb]{1.00,0.00,0.00}{TODO: #1}}}

\newcommand{\argmin}{\operatornamewithlimits{argmin}}
\newtheorem{example}{Example}
\begin{document}
\title{Spatio-Temporal Site Recommendation}
\author{Blinded for Double-Blind review\vspace{2cm}}
%\author{\IEEEauthorblockN{Maximilian Franzke, Tobias Emrich} \IEEEauthorblockA{
%Ludwig-Maximilians-Universit\"at\\
%Munich, Germany\\
%\{franzke, emrich\}@dbs.ifi.lmu.de}
%\and
%\IEEEauthorblockN{Andreas Z\"ufle, Matthias Renz}
%\IEEEauthorblockA{George Mason University\\ Fairfax, VA, USA\\
%\{azufle, mrenz\}@gmu.edu}
%\and
%\IEEEauthorblockN{James Kirk\\ and Montgomery Scott}
%\IEEEauthorblockA{Starfleet Academy\\
%San Francisco, California 96678-2391\\
%Telephone: (800) 555--1212\\
%Fax: (888) 555--1212}
%}
%\author{\IEEEauthorblockN{Maximilian Franzke, Tobias Emrich\\} \IEEEauthorblockA{Institute for Informatics\\
%Ludwig-Maximilians-Universität\\
%Munich, Germany\\
%\{franzke, emrich\}@dbs.ifi.lmu.de}
%\and
%\IEEEauthorblockN{Andreas Züfle, Matthias Renz\\}
%\IEEEauthorblockA{George Mason University, Fairfax, VA, USA   \\
%\{azufle,mrenz\}@gmu.edu}
%}
\maketitle



%\titlenote{(Produces the permission block, and
%copyright information). For use with
%SIG-ALTERNATE.CLS. Supported by ACM.}}
%\subtitle{[Extended Abstract]
%\titlenote{A full version of this paper is available as
%\textit{Author's Guide to Preparing ACM SIG Proceedings Using
%\LaTeX$2_\epsilon$\ and BibTeX} at
%\texttt{www.acm.org/eaddress.htm}}}
%
% You need the command \numberofauthors to handle the 'placement
% and alignment' of the authors beneath the title.
%
% For aesthetic reasons, we recommend 'three authors at a time'
% i.e. three 'name/affiliation blocks' be placed beneath the title.
%
% NOTE: You are NOT restricted in how many 'rows' of
% "name/affiliations" may appear. We just ask that you restrict
% the number of 'columns' to three.
%
% Because of the available 'opening page real-estate'
% we ask you to refrain from putting more than six authors
% (two rows with three columns) beneath the article title.
% More than six makes the first-page appear very cluttered indeed.
%
% Use the \alignauthor commands to handle the names
% and affiliations for an 'aesthetic maximum' of six authors.
% Add names, affiliations, addresses for
% the seventh etc. author(s) as the argument for the
% \additionalauthors command.
% These 'additional authors' will be output/set for you
% without further effort on your part as the last section in
% the body of your article BEFORE References or any Appendices.


\maketitle
\vspace{6cm}
\begin{abstract}
Recommendation systems have become extremely common in recent years, and are utilized in a variety of areas to predict the ``rating'' or ``preference'' that a user would give to a point of interest (PoI), such as a restaurant, a hotel, or a bar. Such systems typically produce a list of recommendations by considering previous ratings of the user, as well as ratings of other users. Not every person rates every point of interest they visit. In this work, we want to explore the use of spatio-temporal data to improve recommendation systems:  We postulate that spatio-temporal user data may indicate the liking or disliking of a point of interest. Clearly, if a user frequently visits the same PoI, stays at the PoI for long times, and is willing to travel a long distance to visit a PoI, that might indicate that user likes that PoI. Thus, we propose to extract user-PoI relation features from spatio-temporal trajectory data only. Using these features, we use out-of-the-box data mining and machine learning solutions, to estimate the popularity of a PoI. Our experimental evaluation shows, that the features extracted from spatio-temporal data able to accurately predict the popularity of a PoI, using ground-truth data from FourSquare as a baseline.
\end{abstract}


%
% The code below should be generated by the tool at
% http://dl.acm.org/ccs.cfm
% Please copy and paste the code instead of the example below.
%
%\begin{CCSXML}
%<ccs2012>
% <concept>
%  <concept_id>10010520.10010553.10010562</concept_id>
%  <concept_desc>Computer systems organization~Embedded systems</concept_desc>
%  <concept_significance>500</concept_significance>
% </concept>
% <concept>
%  <concept_id>10010520.10010575.10010755</concept_id>
%  <concept_desc>Computer systems organization~Redundancy</concept_desc>
%  <concept_significance>300</concept_significance>
% </concept>
% <concept>
%  <concept_id>10010520.10010553.10010554</concept_id>
%  <concept_desc>Computer systems organization~Robotics</concept_desc>
%  <concept_significance>100</concept_significance>
% </concept>
% <concept>
%  <concept_id>10003033.10003083.10003095</concept_id>
%  <concept_desc>Networks~Network reliability</concept_desc>
%  <concept_significance>100</concept_significance>
% </concept>
%</ccs2012>
%\end{CCSXML}
%
%\ccsdesc[500]{Computer systems organization~Embedded systems}
%\ccsdesc[300]{Computer systems organization~Redundancy}
%\ccsdesc{Computer systems organization~Robotics}
%\ccsdesc[100]{Networks~Network reliability}


%
% End generated code
%

%
%  Use this command to print the description
%
%\printccsdesc

% We no longer use \terms command
%\terms{Theory}

%\keywords{graph mining, social networks, attributed graph, non-redundant, overlapping, parameter-free}

\section{Introduction}\label{sec:intro}
Dummy reference \cite{liu2013s}

\section{Related Work}\label{sec:rw}

In a recommender system, there is a set of users and a set of items. The goal of a recommender system is to recommend a user certain items that best match the user's preference. The essential research problem here is, how to predict a user's rating for items that were not rated by him. This is due to the fact that the number of users and items are usually very large, and it is impractical to ask users to rate every item. We study this problem in the context of user-site recommendations.

Two types of recommender systems have been developed in the past few decades. \textbf{Content-based recommender systems} (e.g., \cite{contentbasedLang95,contentbasedPazzani97}) analyze the properties of item (e.g., item descriptions) and/or user profile to identify items that are attractive to the user. The spatial-temporal features we explored is similar to this type of recommendation. Nevertheless, existing methods mostly focus on features extracted from user's interaction with the recommender system (e.g., view or purchase history, reviews, account profile). To our knowledge, this is the first work that explores user's visiting features for user rating prediction. 

The other type is \textbf{Collaborative-filtering recommender systems} (e.g., \cite{userUserRec94,amazonRecommendation,davidson2010youtube,MFRec09})). In general, collaborative-filtering make predictions of users' preference to items based on their similarity to other users. It relies on analysis of large amount existing product rating data. However, it becomes challenging to calculate user-similarity when there is not enough such data, which is known as the cold-start problem. A widely used technique for collaborative-filtering is Matrix factorization~\cite{koren2009matrix}. Matrix factorization works on a user-item rating matrix. It models both users and items as vectors of latent features. Due to space limits, readers are referred to surveys~\cite{RecSurveyTKDE05, RecSysSurvey13} on recommender systems for details. 

Our work is different from Location Recommendation~\cite{ye2010location,wang2013location}, which also considers locations as recommended items. However, they propose to explore geo-social activities of users to facilitate the identification of users with similar preference, which fall in the category of collaborative-filtering. In contrast, our work is to direct predict a user's preference based on his spatial-temporal features.

\section{Problem Definition}
\label{sec:probdef}

\section{Discussion: Stay Point Detection}
%!TEX root = ./main.tex
\section{Spatio-Temporal User-Site Feature Extraction}
\label{sec:method}

There are many spatio-temporal features that could potentially be related to a user's feeling of a location. We discuss in the section a set of common features we select for the rating prediction problem, and the methods we use to extra such features. It is easy to understand why these features are selected since they are intuitive.

\subsection{Extracting the frequency of visits of a user}

Our assumption is, if a user repeatedly visits a PoI, e.g., always dine in the same restaurant, it strongly suggests that the user favours the PoI. On the other hand, if a user visits a PoI only once and never comes back, it suggests the user dislike the place. Therefore we choose the frequency of visits as a feature. Here, each stay-point is considered a visit.

In order to extra frequency information from a user's trajectories, we count for how many time does the user visit a location in a given time period. Typically, each check-in to the location is counted as one visit. Users are then assigned into several frequency groups (Table~\ref{frequencyGroups}) based on how often they visit the location. Note that if a user has never visited a PoI, he cannot rate it. Thus we do not assign any group for such users.

\begin{table}[htbp]
\begin{center}
\caption{Groups based on how often user visits a location \label{frequencyGroups}}
\begin{tabular}{|l|} \hline
\textbf{Frequency groups} \\ \hline
At least one visit per day \\ \hline
At least one visit per week \\ \hline
At least one visit per month \\ \hline
Less than one visit per month \\ \hline
\end{tabular}
\end{center}
\end{table}

We note a user's behaviour may vary over time. For example, if a user likes a shopping center, he may visit the shopping center very frequently during Thanksgiving and similar holidays, while visit the same place only once in several month during the rest of the year. As a result, the timing window used to count visits can have significant impact on the user's group assignment. For the same user, if we consider only visits happened in the Thanksgiving week, the user belongs to the ``At least one visit per week " group. However, if we look at the year-long visits, he may be grouped into ``Less than one visit per month" since his visits are averaged over twelve months. We propose a simple way to mitigate this problem.

First, we use timing windows with different sizes simultaneously to compute the frequency of visits of a user. Specifically, given a user's trajectory data for a period of $n$ days, for each day/week/month within these $n$ days, we count the number of visits that falls in the same day/week/month. We then use this count to compute the user's group of that specific day/week/month. Here we use day/week/month as nature timing windows, but in practice the size of timing window can be arbitrary. Second, when a user's visit frequency demonstrates inconsistency in different time period with the same timing window size, we use his maximal visiting frequency, e.g., if a user visits a location 3 times a week in week 1, but 0 times in week 2, the user is then categorized as ``At least one visit per week" in this case. Given that each user has a unique and consistent ID, for each PoI, we can then calculate how many users falls in each frequency group.

Additionally, for each PoI, we calculate three numerical features that also describe the users' visiting pattern to the PoI: 1) \textbf{Average duration between two consecutive visits of a user}, 2) \textbf{Minimal duration between two consecutive visits of a user}, and 3) \textbf{Maximal duration between two consecutive visits of a user}. These information are supplemental to the frequency groups. We are interested in finding out their impact on the rating prediction result.

\subsection{Extracting the length of stays of a user}

Given a trajectory dataset, the length of stays can be inferred by examining the interval between consecutive check-ins of the user. If two visits are reported within a time threshold $\tau$, e.g., 30-minutes, it is safe to consider the two check-ins belong to the same stay. Thus, the minimal length of this stay is the time frame between the two check-ins. Similarly, an upper bound of the length of stay is the time between a check-in to the location and the closest check-in to a different location. This provides us with a coarse way to estimate the average length of stay of users to a PoI. This method is most accurate if the user's location is reported periodically with small intervals, or whenever the user changes location.

\subsection{Extracting the travel distance of a user from its home base}

If a user is willing to travel a long distance from his home/work area to a PoI, it is likely that the PoI is attractive to him. Unfortunately, measuring the travel distance from one's home/work can be challenging. This is because users' home or work address or coordinates are usually not explicitly marked in a spatial-temporal dataset due to privacy concerns.

We use distance-based location clustering (e.g., weighted k-means~\cite{kerdprasop2005weighted}, where the weight of a location is the number of visits by the user) to estimate a user's home base, i.e., an area where he is likely to live/work. Locations a user visits frequently is likely to formulate a dense cluster in the area where he lives and works~\cite{golle2009anonymity, zang2011anonymization}. If there exists a location that is isolated from any of these clusters, we deem such a location as ``far from home". It requires extra effort for the user to travel this location. Willing to make such effort indicates the user may like the place. An example of home base and isolated locations is illustrated in Figure~\ref{example}.

\begin{figure}[htb]
\center
%\vspace{-2mm}
{\includegraphics[width=0.45 \textwidth]{homebase.pdf}}
%\vspace{-12mm}
\caption{Example of home base and isolated locations} 
\label{example}
%\vspace{-4mm}
\end{figure}

For each isolated location, we estimate the travelling effort for the user to reach the location. Using the distance between the isolated location and the user's home base may be biased. A user who likes to drive can easily travel to a mall several miles away from home, but for someone who does not have a car, travelling the same distance means significantly more effort. Instead, we use the relative travelling distance. For a location $l$, we compute $D_l = d_l / \overline{d_{home}}$ where $d_l$ is the minimal travel distance between $l$ and any home base location, and $\overline{d_{home}}$ the average travel distance between locations within the user's home base.

\subsection{Extracting the type of PoI}

A user's spatial-temporal behaviour can be very different in different types of PoIs. Suppose a user gives high rate for both a coffee stand and a theatre. It is common for a user to visit the coffee stand every day or even multiple times a day, while such a visiting pattern is not likely to appear for his visiting to the theatre. Given the diversity of locations, we believe the type of different PoIs serves as an important feature in the rating prediction process.

Fortunately, in many spatial-temporal datasets (e.g., \cite{yang2014modeling}), a category is given for each PoI. Nevertheless, a dataset may provide only coordinate of visited locations with no additional information. One way to infer the type of PoI in this case is to match the coordinates to PoIs using geoinfo systems such as Google Map, which provides detailed category information of PoIs.


\subsection{Discussion}

Due to limited space, we briefly discuss some other features that could be explored for user rating prediction.

\myparatight{Exploring relation between nearby PoIs}


\myparatight{Exploring social connection between users}


%!TEX root = ./main.tex
\section{User-Site Rating Prediction}
\label{sec:prediction}

1. Simple linear regression - not enough expression power
2. Cov-DNN - Why?
%\input{algorithm}
\section{Experimental Evaluation}\label{sec:exp}

Schemes
1. Linear regression
2. Matrix Factorization - parameters? - Microsoft?
3. Cov-DNN - parameters?

Metrics
1. MSE of prediction
2. Running time(?)
3. 

Dataset
1. NYC
2. Tokyo

Ground truth
1. FourSquare
2. Yelp
3. TripAdvisor

Parameters
1. Training set size / percentage
2. Category of locations
3. NN depth

Data cleaning
1. Being rated by more than 20 users
2. 
\vspace{-0.0cm}
\section{Conclusions}
\label{sec:conclusion}



%
% The following two commands are all you need in the
% initial runs of your .tex file to

\bibliographystyle{abbrv}
\tiny{\bibliography{sigproc} } % sigproc.bib is the name of the Bibliography in this case
% You must have a proper ".bib" file
%  and remember to run:
% latex bibtex latex latex
% to resolve all references
%
% ACM needs 'a single self-contained file'!
%
%APPENDICES are optional
%\balancecolumns
%\appendix
%Appendix A
\end{document}
