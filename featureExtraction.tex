%!TEX root = ./main.tex
\section{Spatio-Temporal User-Site Feature Extraction}
\label{sec:method}

There are many spatio-temporal features that could potentially be related to a user's feeling of a location. We discuss in the section a set of common features we select for the rating prediction problem, and the methods we use to extra such features. It is easy to understand why these features are selected since they are intuitive.

\subsection{Extracting the frequency of visits of a user}

Our assumption is, if a user repeatedly visits a place, e.g., always dine in the same restaurant, it strongly suggests that the user would rate this place highly. On the other hand, if a user visits a place only once, then never comes back, it may be because the user dislike the place. Therefore we choose the frequency of visits as a feature.

In order to extra frequency information from user trajectories, for a location and a user who visited it, we count for how many time does the user visit the location in a given time period. Users are then assigned into several frequency groups (Table~\ref{frequencyGroups}) based on how often they visit the location.

\begin{table}[htbp]
\begin{center}
\caption{Groups based on how often user visits a location \label{frequencyGroups}}
\begin{tabular}{|l|} \hline
\textbf{Frequency groups} \\ \hline
At least one visit per day \\ \hline
At least one visit per week \\ \hline
At least one visit per month \\ \hline
Less than one visit per month \\ \hline
Never visited \\ \hline
\end{tabular}
\end{center}
\end{table}

We note a user's behaviour may vary over time. For example, a user rates a shopping center highly. He may visit the shopping center very frequently during Thanksgiving, while visit the same place only once in several month during the rest of the year, because he does not have much to buy. As a result, the timing window used to count visits can have a significant impact on the user's group assignment. If we consider only visits happened in the Thanksgiving week, the user belongs to the ``At least one visit per week " group. However, if we look at the year-long visits, the same user may be grouped into ``Less than one visit per month" since his visits are averaged over twelve months. We propose a simple way to mitigate this problem.

First, we use timing windows with different sizes simultaneously to compute the frequency of visits of a user. Specifically, given a user's trajectory data for a period of $n$ days, for each day/week/month within these $n$ days, we count the number of visits that falls in the same day/week/month. We then use this count to compute the user's group of that specific day/week/month. Here we use day/week/month as nature timing windows, but in practice the size of timing window can be arbitrary. Second, when a user's visit frequency demonstrates inconsistency in different time period with the same timing window size, we use his maximal visiting frequency, e.g., if a user visits a location 3 times a week in week 1, but 0 times in week 2, the user is then categorized as ``At least one visit per week" in this case.

Additionally, for each user, we calculate three numerical features that also describe the user's visiting pattern to a location: 1) \textbf{Average duration between two consecutive visits}, 2) \textbf{Minimal duration between two consecutive visits}, and 3) \textbf{Maximal duration between two consecutive visits}. These information are supplemental to the frequency groups. We are interested in finding out their impact on the rating prediction result.

\subsection{Extracting the length of stays of a user}




\subsection{Extracting the travel distance of a user from its home base}

If a user is willing to travel a long distance from his home/office area to a restaurant, it is very likely that the restaurant is attractive to him. Unfortunately, users' home or work address are usually not included in the dataset due to privacy concerns. 

We use distance-based location clustering (e.g., \cite{} ) to estimate a user's home base, i.e., an area where he is likely to live/work. We assume, naturally, locations a user visits frequently formulates a large cluster in the area where he lives. If there exists a location that is isolated from any of these clusters, we deem such a location as ``far from home" and thus it requires extra effort for the user to travel there. Willing to make such effort indicates the user may like the place. And we measure this effort using the minimal distance between the location and estimated home base of the user. An example of home base and far locations is illustrated in Figure~\ref{example}.

\begin{figure}[htb]
\center
%\vspace{-2mm}
{\includegraphics[width=0.45 \textwidth]{homebase.pdf}}
%\vspace{-12mm}
\caption{Example of home base and ``isolated" locations} 
\label{example}
%\vspace{-4mm}
\end{figure}

Using cluster weight to decide if home base or not.

\subsection{Extracting the total number of different visitors}

We observe that in reality, a highly-rated location naturally attracts many different users.


\subsection{Extracting the type of location}


\subsection{Discussion}
1. Assumptions: when are valid and when is not
2. Lack of data: 
3. Explore social connection
4. Compare nearby PoIs
