\section{Discussion: Stay Point Detection} 
The first step of our user-site-recommendation approach requires to map a raw trajectory $\DB(u)$ of a user $u$ to a stay-trajectory $(\DB\bowtie\mathcal{S})(u)$. Such stay points can be detected by using existing work such as proposed in \cite{li2008mining,zheng2009mining,zheng2010geolife,xiao2010finding}. All of these works use a distance threshold $\theta_{d}$ and defines a stay as the duration of time where the trajectory does not exceed a distance of $\theta_{d}$ to a PoI. In this work, we entirely circumvent the step of implicit stay point detection, by using data that has explicit stay points. Therefore, in our experimental evaluation we use Check-in data from location based social networks (LBSNs), which explicitly contain the stay points of users. Clearly, by circumventing the problem of stay point detection, we limit our experimental evaluation to LBSN Check-in data, which is relatively small (hundreds of megabytes of data), compared to large raw trajectory databases (Terrabytes of data).
Yet, we postulate that our relatively small FourSquare Check-in dataset \cite{?}, allows to effectively predict the user rating of a restaurant. This hypothesis is also supported by our experiments.

While we circumvent the problem of stay point detection by using Check-in data, we still need to estimate the duration of stay, as the duration of a stay is one of the features that we will use for prediction in Section \ref{?}. Thus, we interpret each Check-in $c$ as a stay point. If the corresponding user has no other check-ins for the next six hours, we set the duration of $c$ to \emph{UNKNOWN}. Otherwise, the set the duration to the time in-between these two check-ins.   